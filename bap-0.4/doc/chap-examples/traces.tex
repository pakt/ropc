\section{Traces}

In this section, we will explain how to use our PIN-based trace
recording tool to record a trace.  We then show how to perform common
trace analysis tasks.

\subsection{Setup}

The PIN trace tool is located in the \cmdline{pintraces/}
directory.  Because of licensing reasons, we cannot distribute PIN
with the trace tool.  PIN must be extracted inside of the traces
directory; for instance, if the traces branch is located in
\cmdline{\$PATH/traces}, then PIN should be in
\cmdline{\$PATH/traces/pin}.  Running ./getpin.sh from the
\cmdline{\$PATH/traces/pintraces} directory will automatically
download and install PIN for Linux; the user is responsible for
accepting the PIN license agreements.  On Windows, PIN must be
downloaded and extracted by hand.  The PIN tool can be built by
executing \cmdline{make} in the \cmdline{\$PATH/traces/experiemental}
directory.  On Windows, we reccomend using GNU Make for Windows
(\url{http://gnuwin32.sourceforge.net/packages/make.htm}) and Visual
C++ 2010 express
(\url{http://www.microsoft.com/visualstudio/en-us/products/2010-editions/visual-cpp-express}).
After compilation, the PIN tool should exist in \\
\cmdline{\$PATH/traces/pintraces/obj-ia32/gentrace.so} (or
gentrace.dll on Windows).  In the rest of the chapter, we will assume
Linux is being used; most interaction with the trace tool is the same.

\subsection{Recording a trace}

To see the command line options to the trace tool, execute 

\begin{verbatim} 
$PATH/traces/pin/pin -t \
 $PATH/traces/pintraces/obj-ia32/gentrace.so -h -- /bin/ls
\end{verbatim}

By default, the trace tool will only log instructions that are
\emph{tainted}, i.e., those that depend on user input.  The options
that begin with taint are used to mark various data
as being user input.  For instance, -taint-files readme marks the file
readme as being user input.

We will record a trace of a simple buffer overflow.  Run 

\begin{verbatim}
echo ``helloooooooooooooooooo'' > readme
\end{verbatim}

 to create the input file.  Then run 

\begin{verbatim}
make $PATH/traces/pintraces/examples/bof1/
\end{verbatim}

and

\begin{verbatim}
$PATH/traces/pin/pin -t  \
 $PATH/traces/pintraces/obj-ia32/gentrace.so -taint-files readme \
 -- $PATH/traces/pintraces/examples/bof1/bof1
\end{verbatim}

The PIN tool will output
many debugging messages; this is normal.  If the trace tool detected
the buffer overflow, it will print ``stack smashing detected'' near
the end of the logs.  At this point, there should be a trace file
ending with suffix bpt in the current working directory.  In the
following commands, we assume this file is named trace.bpt.

To lift the trace data and print it, run 

\begin{verbatim}
iltrans -pin -trace trace.bpt -pp-ast /dev/stdout
\end{verbatim}

It is also possible to concretize the trace, which removes jumps and performs
memory concretization, by executing 

\begin{verbatim}
iltrans -pin -trace trace.bpt -trace-concrete -pp-ast /dev/stdout
\end{verbatim}

Adding the -trace-check option before -trace-concrete causes BAP to compare its
internal evaluator's notion of state with the actual values recorded in the
trace.  It can be used to check for bugs in the IL.  Finally, running 

\begin{verbatim}
iltrans -pin -trace trace.bpt -trace-formula f
\end{verbatim}

will symbolically execute the trace and output the generated verification
condtion to the file f. This can then be solved with stp to find satisfying
answers to the trace.
