\section{Tools from \bil}

\subsection{{\tt topredicate}}

{\tt topredicate} supports program verification in two ways. First,
{\tt topredicate} can convert the IL into Dijkstra's Guarded Command
Language (GCL), and calculate the weakest precondition with respect to
GCL programs~\cite{dijkstra:1976}. The weakest precondition for a
program with respect to a predicate $q$ is the most general condition
such that any input satisfying the condition is guaranteed to
terminate (normally) in a state satisfying $q$.  Currently we only
support acyclic programs, i.e., we do not support GCL {\tt while}.

{\tt topredicate} also interfaces with decision procedures.  {\tt
  topredicate} can write out expressions (e.g., weakest preconditions)
in CVC Lite syntax~\cite{cvclite}, which is supported by several
decision procedures. In addition, {\tt topredicate} interfaces
directly with the STP~\cite{ganesh:2007} decision procedure through
calls from \bap to the STP library.



\subsection{{\tt ileval}} 

{\tt ileval} evaluates a given \bil
program using the operational semantics in in~\ref{vine:operational}.
The evaluator allows us to execute programs without recompiling the IL
back down to assembly. For example, we can test that a raised program
is correct by executing the IL on an input $i$, observing the value $v$,
executing the original binary program on $i$, observing the value
$v'$, and verifying $v = v'$.

\subsection{{\tt toc}} 

{\tt toc} generates valid C code from the IL.  For example, one could
use this as a rudimentary decompiler by first raising assembly to
\bil, then writing it out as valid C.  The ability to export to C also
provides a way to compile \bil programs: the IL is written as C, then
compiled with a C compiler.

The C code generator implements memories in the IL as arrays.  A {\tt
  store} operation is a store on the array, and a {\tt load} is a load
from the array.  Thus, C-generated code simulates real memory.  For
example, consider a program that is vulnerable to a buffer overflow
attack. It is raised to \bil, then written as C and recompiled. An
out-of-bound write on the original program will be simulated in the
corresponding C array, but will not lead to a real buffer overflow.

